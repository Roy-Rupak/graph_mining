\section{Introduction}
\label{sec:introduction}

Graph representation of huge dataset used for solving real world problems
is a popular format to feed computable data to the datamining applications.
Oftentimes, these data structures can be complicated with respect to its
connectivity among the elements of the representation. This dependency
among the datapoints or nodes of the graph not only makes it a complex task
to extract the insights from the data through datamining, but also renders
the application workflow as a hindrance to the parallelization for
efficiently leveraging high performance computing (HPC) infrastructures.
The goal of the project is to perform research on applying clustering
techniques on large graph representation of huge real world dataset
in order to maximize data parallelism in the datamining models for
utilizing HPC resources while ensuring reasonable accuracy of the system.

In particular, the research intends to initialize satisfying the general
motivation of applying graph clustering to improve parallelism
by focusing into a special and narrow sector as a proof of concept.
This sector deals with feeding large geographic map data to a system
for preprocessing and formatting the data in graph representation
where a coordinate, i.e., tuple of longitude and lattitude of
a specific geographic point, is treated as a node and the connecting
road between two coordinates is presented as the edge.
This large graph presenting the map can be used by the system to
solve different problems in the transportation system. For example,
this data can be used for finding the shortest path between two points of the graph.

Using legacy algorithms like Dijkstra's algorithm or even comparatively
more modern techniques like A* search will pose
immense time complexity if run on sequential systems.
Performance maximization via using distributed HPC systems can only be possible
if divide and conquer strategy can be applied on the data representation
by clustering the graph into smaller chunks and run the algorithm
upon the lower volume data in parallel, and finally combining
all the small results to deduce the final outcome.

Taking all these research requirements into consideration, this project proposes the following contributions.
\begin{itemize}
	\item A thorough survey on the state-of-the-art graph clustering techniques in general.
	\item Application of different graph clustering on huge geographic data as a special case handling.
	\item Execution of a shortest path algorithm as a proof of concept for evaluating the efficacy of different clustering techniques.
	\item Verification of performance and accuracy of the application for different clustering on HPC systems via thorough experiments.
	\item Comparison among different clustering techniques and recommendation for the specific data workload in concern, i.e. geographic map data.
\end{itemize}
