\section{Conclusion and Future Work}
With the easy representability, graph has become a popular data structure for simplifying real world computation intensive problems. These real world problems are usually sequential and difficult to accelerate using modern distributed HPC systems. This work addresses graph parallelization problem, with the use of clustering approach by incorporating a master-slave architecture with the traditional single source shortest path algorithm. Our approach ensures data parallel execution of the algorithm in different partitions of the entire graph with a view to making it suitable for modern HPC systems. We have made a comparative analysis between partitioned-based and hierarchical clustering when applied on graph data structure. Our experiments show that hierarchical agglomerative clustering outperforms kmeans clustering in terms of accuracy and latency for graph parallelization. Due to the high connectivity among the different clusters of the graph data, the final reduction process to generate final output from the sub-outputs from each clusters is a challenging task even for the cutting edge HPC systems. There are still some scopes of improvement which we would like to leave as future work. Extending the existing clustering algorithms to reduce the inter-cluster edges can help us accelerate the proposed algorithm with better accuracy. Moreover, optimizing the gateway detection algorithm further can alleviate the latency degradation caused due to shortest path merging phase. Evaluating our approach for heuristic based algorithms, e.g., A* search, can also be an interesting line of study.
\label{sec:conclusion}