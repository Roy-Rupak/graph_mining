\section{Literature Survey}
\label{sec:literature_survey}

Clustering the graph representation for geographical dataset continues to gain 
huge attention as it shows its potential for solving many real world problems. 
However, time taken by the graph algorithms is still an important challenge for 
real-time analysis which can be minimized by using the HPC infrastructure.
In~\cite{parallel_graph_algorithm} Bogle et. al. presents a new 
distributed-memory, BFS based parallel algorithm to identify and remove 
disconnected components or hinge vertices for graph-algorithms while 
significantly improving the convergence of iterative solvers. This work is 
extremely related in our context as the objective is to develop an efficient 
and accurate parallel algorithm for geographical dataset using its graphical 
representation.

Clustering the graphs in each node plays an important role in solving this 
scenario and state of the art researches 
need to be explored in this regard. Efficient semi-supervised learning can be 
one of the choice while doing the graph 
clustering.
Dhillon et. al.~\cite{kernel_kmeans} shows semi-supervised learning can improve 
the  graph 
clustering results and states an objective function for weighted kernel k-means 
by using Hidden Markov Random Fields, certain class of constraint penalty 
function and squared Euclidean Distance. 
Kulis et. al.~\cite{semi_supervised_graph_clustering} proposes SS-KERNELKMEANS 
to optimize a semi-supervised clustering objective for graph-base inputs by 
proposing a more generalized formulation and also exhibits an equivalence 
between the kernel k-means objective function
and a special case of the HMRF-based
semi-supervised clustering objective.

Detecting densely connected groups in a large graph is also a primary 
requirement in our approach. Zhou et. 
al.~\cite{structural_attribute_similarity_clustering} proposes SA-Cluster, a 
novel graph clustering algorithm based on both structural and attribute 
similarities using a unified distance measure. This method considers the 
heterogeneous vertex properties apart from the overall topological structure 
for clustering and shows its effectiveness both theoritically and 
experimentally.

Even the possibilities of using deep learning in graph clustering is explored 
by Tian et. al.~\cite{deep_representation_graph_clustering} and the work 
illustrates that it has strong theoritical foundation. The autoencoder based 
approach used in this paper also outperforms conventional spectral clustering 
with much lower computational complexity.
Bruna et. al.~\cite{community_detection} investigates a generic family of graph 
neural networks and shows data-driven approaches to clustering can be used with 
real dataset. Even it shows that the data-driven approach can consume less 
computational steps while performing better than other rigid parametric models.
