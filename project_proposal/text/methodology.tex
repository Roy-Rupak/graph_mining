\section{Methodology and Algorithm}
\label{sec:methodology}

This section discusses the characteristics and source of geographic map data,
tentative clustering techniques and algorithms to find the shortest
path between two data points.

\subsection{Geographic Dataset}
The data for this project will be collected from OpenStreetMap~\cite{osm},
which is an open source platform for maintaining map data of the entire earth.
There are several sources of getting the data, e.g., extracting directly from
the website for a small region or the entire planet, or continent, country or
state based data from Geofabrik Download Server~\cite{geo_dl_server}, etc.
Data normally comes in the human-readable XML formatted .osm files.
A basic format of the data presenting a datapoint by
$node$ and a connected path by $way$ XML DOM element
is demonstrated in Appendix~\ref{app:osm_data}.

The $node$ and $way$ information can be used to construct a weighted adjacency
matrix denoting the entire map, where weight of an edge is the distance between
two nodes.
In the preprocessing stage, if the whole adjacency matrix is too large for
the system memory, this can be written to a file and kept in storage devices.
Later, according to the application performance requirements, density of the
data can be reduced by only keeping the vertices that are the intersection
points of three or more edges and update weight accordingly,
because intersection of two edges actually will denote a single road.

\subsection{Clustering Techniques}

One of the most vital goals of this project is to perform research on different
clustering techniques starting from very basic clustering like K-means
and hierarchical clustering to advanced techniques discussed in
Sec.~\ref{sec:state_of_the_art}.~\cite{structural_attribute_similarity_clustering, deep_representation_graph_clustering, parallel_graph_algorithm}

Specifically, some assumptions regarding transportation network can be leveraged
to set as constraints in the clustering techniques in order to acquire
faster convergence and ensure higher parallelism via lower connectivity or dependency among
the clusters. As an initial stage thought, the total graph should be clustered
in such a way that the clusters hold the following properties:
\begin{itemize}
	\item There are some common vertices (one or more than one) between the clusters named $gateways$.
	\item The number of nodes in a cluster cannot exceed a maximum value.
	\item The $gateways$ between two clusters cannot be connected to each other.
\end{itemize}

\subsection{Algorithm Implementation Plan}

This section discusses the important data structures, heuristics,
tentative algorithms and implementation plans.

\subsubsection{Important Data Structures}

The data structures stated in Appendix~\ref{app:data_structure} can be used in 
the algorithms to denote a node and a cluster.

\subsubsection{Associated Heuristics}

Let us consider a node $n$,
\\Real distance from source node to $n$,
$g(n) = RealDist(source, n)$
\\Eucledian distance from $n$ to destination node
$h(n) = EucledianDist(n, dest)$
\\Heuristic function of $n$,
$f(n) = g(n) + h(n)$
\\Heuristic value of a cluster $c$,
$c = {n_1, n_2, ..., n_n}$
$fc(c) = mean(f(n_1), f(n_2), ..., f(n_n))$

\subsubsection{Tentative Shortest Path Parallel Algorithm}

Using the above heuristics, the algorithm for finding a shortest
path between two vertices after clustering can be devised as follows:

\begin{algorithm}[H]
    \label{alg:shortest_path_with_heuristics}
    \caption{Find Shortest Path with Heuristics}
    \begin{algorithmic}[1]
    \renewcommand{\algorithmicrequire}{\textbf{Input:}}
    \renewcommand{\algorithmicensure}{\textbf{Output:}}
    \REQUIRE graph, source, dest
    \ENSURE  path

    \STATE $base\_dist = EucledianDist(source, dest)$
    \STATE $dist = EucledianDist(curr\_node, dest)$
    \IF {($dist < base\_dist$)}
    \STATE Ignore the node
    \ENDIF
    \STATE $g(curr\_node) = RealDist(source, curr\_node)$
    \STATE $h(curr\_node) = EucledianDist(curr\_node, dest)$
    \STATE $f(curr\_node) = g(curr\_node) + h(curr\_node)$
    \STATE Visit the next node that has lowest $f(curr\_node)$ value
    \RETURN shortest $path$ found by the shortest path search algorithm
    \end{algorithmic}
\end{algorithm}
\begin{algorithm}[H]
    \label{alg:shortest_path_with_clustering}
    \caption{Find Shortest Path with Clustering}
    \begin{algorithmic}[1]
    \renewcommand{\algorithmicrequire}{\textbf{Input:}}
    \renewcommand{\algorithmicensure}{\textbf{Output:}}
    \REQUIRE graph, source, dest, cluster\_list
    \ENSURE  path

    \IF {($source.cluster\_id == dest.cluster\_id$)}
    \STATE $Find Shortest Path With Heuristics$
    \STATE $(cluster\_list[cluster\_id], source, dest)$
    \ELSE
    \STATE Construct $graph_c$ representing the clusters as nodes and the $gateways$ as edges
    \STATE weights of the edges = heuristics of the ​$gateway$ nodes​
    \STATE heuristic value of each cluster = mean(heuristics of all nodes in cluster)​
    \STATE $Find Shortest Path With Heuristics$
    \STATE $(graph_c, source_c, dest_c)$
    \STATE Apply ​$Find Shortest Path With Heuristics$ recursively in every cluster in that sequence to find the final path​
    \ENDIF
    \RETURN shortest $path$ found by the shortest path search algorithm
    \end{algorithmic}
\end{algorithm}
